\documentclass{beamer}
%\usepackage[latin1]{inputenc}
\usetheme{Warsaw}
\title[Intro to Python: Week 1]{Introduction  to Python\\
Lambda, and Functional Programming}
\author{Christopher Barker}
\institute{UW Continuing Education}
\date{November 5, 2013}

\usepackage{listings}
\usepackage{hyperref}

% \renewcommand{\baselinestretch}{1.2} 

\begin{document}

% ---------------------------------------------
\begin{frame}
  \titlepage
\end{frame}

% ---------------------------------------------
\begin{frame}
\frametitle{Table of Contents}
%\tableofcontents[currentsection]
  \tableofcontents
\end{frame}


\section{Review/Questions}

% ---------------------------------------------
\begin{frame}{Review of Previous Class}

\begin{itemize}
  \item Unicode?
  \item Keyword arguments?
  \item Comprehensions?
  \item Unit testing?
\end{itemize}

\end{frame}


% ---------------------------------------------
\begin{frame}{Lightning Talks}

\vfill
{\LARGE Lightning talks today:}

\vfill
{\Large
Lawrence Chan

\vfill
Kimberly Colwell

\vfill
Maria Petrova


}
\vfill

\end{frame}


% ---------------------------------------------
\begin{frame}{Homework review}

  \vfill
  {\Large Homework Questions? }

  \vfill

\end{frame}

%########################
\section{Lambda}

% ---------------------------------------------
\begin{frame}[fragile]{lambda}

{\Large``Anonymous'' functions}

\vfill
\begin{verbatim}
In [171]: f = lambda x, y: x+y

In [172]: f(2,3)
Out[172]: 5
\end{verbatim}

\vfill
{\Large Can only be an expression -- not a statement}

\end{frame} 

% ---------------------------------------------
\begin{frame}[fragile]{lambda}

{\Large Called ``Anonymous'': it doesn't need a name.}

\vfill
{\Large It's a python object, it can be stored in a list or other container}

\vfill
\begin{verbatim}
In [7]: l = [lambda x, y: x+y]
In [8]: type(l[0])
Out[8]: function
\end{verbatim}

\vfill
{\Large And you can call it:}

\vfill
\begin{verbatim}
In [9]: l[0](3,4)
Out[9]: 7
\end{verbatim}

\end{frame} 


% ---------------------------------------------
\begin{frame}[fragile]{functions as first class objects}

{\Large You can do that with ``regular'' functions too:}

\vfill
\begin{verbatim}
In [12]: def fun(x,y):
   ....:     return x+y
   ....: 

In [13]: l = [fun]

In [14]: type(l[0])
Out[14]: function

In [15]: l[0](3,4)
Out[15]: 7
\end{verbatim}

\end{frame} 

\section{Functional Programming}

% ---------------------------------------------
\begin{frame}[fragile]{map}

{\Large \verb|map| ``maps'' a function onto a sequence of objects --
It applies the function to each item in the list, returning another list}

\begin{verbatim}
In [23]: l = [2, 5, 7, 12, 6, 4]
In [24]: def fun(x):
             return x*2 + 10

In [25]: map(fun, l)
Out[25]: [14, 20, 24, 34, 22, 18]
\end{verbatim}

{\Large But if you only need that function once:}
\begin{verbatim}
In [26]: map(lambda x: x*2 + 10, l)
Out[26]: [14, 20, 24, 34, 22, 18]
\end{verbatim}

\end{frame} 

% ---------------------------------------------
\begin{frame}[fragile]{filter}

{\Large \verb|filter| ``filters'' a sequence of objects with a boolean function -- 
It keeps only those for which the function is True
}

\vfill
{\Large To get only the even numbers}

\begin{verbatim}
In [27]: l = [2, 5, 7, 12, 6, 4]

In [28]: filter(lambda x: not x%2, l)
Out[28]: [2, 12, 6, 4]
\end{verbatim}

\end{frame} 

% ---------------------------------------------
\begin{frame}[fragile]{reduce}

{\Large \verb|reduce| ``reduces'' a sequence of objects to a single object with a function that combines two arguments}

\vfill
{\Large To get the sum:}

\begin{verbatim}
In [30]: l = [2, 5, 7, 12, 6, 4]

In [31]: reduce(lambda x,y: x+y, l)
Out[31]: 36
\end{verbatim}

{\Large To get the product:}

\begin{verbatim}
In [32]: reduce(lambda x,y: x*y, l)
Out[32]: 20160
\end{verbatim}

\end{frame} 

% ---------------------------------------------
\begin{frame}[fragile]{comprehensions}

{\Large Couldn't you do all this with comprehensions?}

\vfill
{\LARGE Yes:}
\begin{verbatim}
In [33]: [x+2 + 10 for x in l]
Out[33]: [14, 17, 19, 24, 18, 16]

In [34]: [x for x in l if not x%2]
Out[34]: [2, 12, 6, 4]
\end{verbatim}

{\Large Except Reduce}

\vfill
But Guido thinks almost all uses of reduce are really \verb|sum()|
\end{frame} 

% ---------------------------------------------
\begin{frame}[fragile]{functional programming}

\vfill
{\Large Comprehensions and map, filter, reduce are all ``functional programming'' approaches}

\vfill
{\Large \verb|map, filter| and \verb|reduce| pre-date comprehensions in Python's history}

\vfill
{\Large Some people like that syntax better}

\vfill
{\Large And ``map-reduce'' is a big concept these days for parallel processing of ``Big Data'' in NoSQL databases.}

\vfill
{\Large (Hadoop, MongoDB, etc.) }

\end{frame} 


% ---------------------------------------------
\begin{frame}[fragile]{lambda}

{\Large Can also use keyword arguments}

\begin{verbatim}
In [186]: l = []
In [187]: for i in range(3):
    l.append(lambda x, e=i: x**e)
   .....:     
In [189]: for f in l:
    print f(3)
1
3
9
\end{verbatim}

{\Large Note when the keyword argument is evaluated: this turns out to be handy}

\end{frame} 
%-------------------------------
\begin{frame}[fragile]{LAB}

{\large
\vfill
\begin{itemize}
  \item Write a function that returns a list of n functions,
such that each one, when called, will return the input value,
incremented by an increasing number.

  \item Use a for loop, \verb|lambda|, and a keyword argument
\end{itemize}

\vfill
\verb|code/lambda/lambda_keyword.html(rst)| \\
\verb|code/lambda/lambda_keyword.py| \\
\verb|code/lambda/test_lambda_keyword.py| \\
}

\end{frame}



%-------------------------------
\begin{frame}{Lightning Talks}

{\LARGE Lightning Talks:}

{\l\Large
\vfill
Lawrence Chan

\vfill
Kimberly Colwell

}
\end{frame}

%%%%%%%%%%%%%%%%%%%%%%%%%%%%%%%%%%%%%%%%%
\section{Object Oriented Programming}

% ---------------------------------------------
\begin{frame}[fragile]{Object Oriented Programming}

\vfill
 {\Large More about Python implementation than OO design/strengths/weaknesses}

\vfill
{\Large One reason for this:\\
Folks can't even agree on what OO ``really`` means}

\vfill
The Quarks of Object-Oriented Development - Deborah J. Armstrong:\\
\url{http://agp.hx0.ru/oop/quarks.pdf}

\end{frame} 

% ---------------------------------------------
\begin{frame}[fragile]{Object Oriented Programming}

\vfill
 {\LARGE Is Python a ``True'' Object-Oriented Language?}

\vfill
{\Large (Doesn't support full encapsulation, doesn't require
objects, etc...)}

\end{frame} 

% ---------------------------------------------
\begin{frame}[fragile]{Object Oriented Programming}

\vfill
 {\LARGE I don't Care!}

\vfill
{\Large Good software design is about code re-use, clean separation of concerns,
refactorability, testability, etc...}

\vfill
{\Large OO can help with all that, but:
\begin{itemize}
  \item It doesn't guarantee it
  \item It can get in the way
\end{itemize}
}

\end{frame} 

% ---------------------------------------------
\begin{frame}[fragile]{Object Oriented Programming}

\vfill
 {\LARGE Python is a Dynamic Language}

\vfill
{\Large That clashes with ``pure'' OO}

\vfill
{\Large Think in terms of what makes sense for your project

 -- not any one paradigm of software design.
}
\end{frame} 

% ---------------------------------------------
\begin{frame}[fragile]{Object Oriented Programming}

\vfill
 {\LARGE OO for this class:}

\vfill
{\Large 
``Objects can be thought of as wrapping their data \\[.03in]
within a set of functions designed to ensure that \\[.03in]
the data are used appropriately, and to assist in \\[.03in]
that use``
}

\vfill
{\small
\url{http://en.wikipedia.org/wiki/Object-oriented_programming}
}

\end{frame} 

% ---------------------------------------------
\begin{frame}[fragile]{Object Oriented Programming}

\vfill
{\LARGE Even simpler:}

\vfill
{\Large 
Objects are data and the functions that act on them in one place.
}

\vfill
{\Large 
In Python: just another namespace.
}
\end{frame} 

% ---------------------------------------------
\begin{frame}[fragile]{Object Oriented Programming}

\vfill
{\LARGE The OO buzzwords:

\vfill
\begin{itemize}
  \item data abstraction
  \item encapsulation
  \item modularity
  \item polymorphism
  \item inheritance
\end{itemize}
}
\end{frame} 

% ---------------------------------------------
\begin{frame}[fragile]{Object Oriented Programming}

\vfill
{\LARGE You can do OO in C}

(see the GTK+ project)

\vfill
{\Large 
``OO languages'' give you some handy tools to make it easier (and safer):
}

\vfill
{\Large
\begin{itemize}
  \item polymorphism (duck typing gives you this anyway)
  \item inheritance
\end{itemize}
}
\end{frame} 

% ---------------------------------------------
\begin{frame}[fragile]{Object Oriented Programming}

\vfill
{\Large OO is the dominant model for the past couple decades

\vfill
You will need to use it:

\vfill
-- It's a good idea for a lot of problems

\vfill
-- You'll need to work with OO packages
}
\end{frame} 

% ---------------------------------------------
\begin{frame}[fragile]{Object Oriented Programming}

\vfill
{\LARGE Some definitions}

\begin{description}
  \item[class] A category of objects: particular data and behavior: A ``circle'' (same as a type in python)
  \item[instance] A particular object of a class: a specific circle
  \item[object] The general case of a instance -- really any value\\ (in Python anyway)
  \item[attribute] Something that belongs to an object (or class)
    -- generally thought of as a variable, or single object, as opposed to a ...
  \item[method] A function that belongs to a class
\end{description}

\end{frame} 


%%%%%%%%%%%%%%%%%%%%%%%%%%%%
\section{Python Classes}

% ---------------------------------------------
\begin{frame}[fragile]{Python Classes}

{\Large The \verb|class| statement}

\vfill
{\large \verb|class| creates a new type object:}

\begin{verbatim}
In [4]: class C(object):
    pass
   ...: 
In [5]: type(C)
Out[5]: type
\end{verbatim}

{\large It is created when the statement is run -- much like \verb|def|}

\vfill
(note on``new style'' classes)

\end{frame} 

% ---------------------------------------------
\begin{frame}[fragile]{Python Classes}

{\Large Note about the book (TP):}

Chapters 15 and 16 use a style that generally isn't recommended:

\begin{verbatim}
In [6]: class Point(object):
   ...:     pass
In [7]: p = Point()
In [8]: p.x = 4
In [9]: p.y = 2
\end{verbatim}

Python is Dynamic -- you can do this, but you generally want more structure,
defaults, etc. 

\vfill
(it used to be a quick and dirty "struct"\\
\hspace{0.2in} -- but use a named tuple now)
\end{frame} 

% ---------------------------------------------
\begin{frame}[fragile]{Python Classes}

{\Large About the simplest class:}

\begin{verbatim}
>>> class Point(object):
...     x = 1
...     y = 2
>>> Point
<class __main__.Point at 0x2bf928>
>>> Point.x
1
>>> p = Point()
>>> p
<__main__.Point instance at 0x2de918>
>>> p.x
1
\end{verbatim}
\end{frame} 

% ---------------------------------------------
\begin{frame}[fragile]{Python Classes}

{\Large Basic Structure of a real class:}

\begin{verbatim}
class Point(object):
# everything defined in here is in the class namespace
    def __init__(self, x, y):
        self.x = x
        self.y = y
## create an instance of that class        
p = Point(3,4)

## access the attributes
print "p.x is:", p.x
print "p.y is:", p.y
\end{verbatim}

{\large see: \verb|code/simple_class| }

\end{frame} 


% ---------------------------------------------
\begin{frame}[fragile]{Python Classes}

{\LARGE The Initializer}

\vfill
{\Large The \verb|__init__| special method is called when a new instance of a class is created.}

\vfill
{\Large You can use it to do any set-up you need}

\vfill
\begin{verbatim}
class Point(object):
    def __init__(self, x, y):
        self.x = x
        self.y = y
\end{verbatim}
\vfill
{\Large It gets the arguments passed to the class constructor}
\end{frame} 

% ---------------------------------------------
\begin{frame}[fragile]{Python Classes}

{\LARGE \verb|self|}

\vfill
{\Large The instance of the class is passed as the first parameter for every method.}

\vfill
{\Large ``\verb|self|'' is only a convention -- but you DO want to use it.}

\vfill
\begin{verbatim}
class Point(object):
    def a_function(self, x, y):
...
\end{verbatim}
\vfill
{\Large Does this look familiar from C-style procedural programming?}
\end{frame} 


% ---------------------------------------------
\begin{frame}[fragile]{Python Classes}

\begin{verbatim}
class Point(object):
    def __init__(self, x, y):
        self.x = x
        self.y = y
\end{verbatim}

\vfill
{\Large Anything assigned to a \verb|self.| attribute is kept in the instance
name space}

\vfill
{\Large That's where all the instance-specific data is.}

\vfill
\end{frame} 

% ---------------------------------------------
\begin{frame}[fragile]{Python Classes}

\begin{verbatim}
class Point(object):
    size = 4
    color= "red"
    def __init__(self, x, y):
        self.x = x
        self.y = y
\end{verbatim}

\vfill
{\Large Anything assigned in the class scope is a class attribute -- every
instance of the class shares the same one.}
\vfill
\end{frame} 

% ---------------------------------------------
\begin{frame}[fragile]{Python Classes}

\begin{verbatim}
class Point(object):
    size = 4
    color= "red"
...
    def get_color():
        return self.color

>>> p3.get_color()
 'red'
\end{verbatim}

\vfill
{\Large class attributes are accessed with \verb|self| also..} 
\vfill
\end{frame} 


% ---------------------------------------------
\begin{frame}[fragile]{Python Classes}

{\Large Typical methods}
\begin{verbatim}
class Circle(object):
    color = "red"
    def __init__(self, diameter):
        self.diameter = diameter

    def grow(self, factor=2):
        self.diameter = self.diameter * factor
\end{verbatim}

\vfill
{\Large methods take some parameters, manipulate the attributes in \verb|self|} 

\end{frame} 

% ---------------------------------------------
\begin{frame}[fragile]{Python Classes}

{\Large Gotcha!}
\begin{verbatim}
...
    def grow(self, factor=2):
        self.diameter = self.diameter * factor
...
In [205]: C = Circle(5)
In [206]: C.grow(2,3)

TypeError: grow() takes at most 2 arguments (3 given)
\end{verbatim}

\vfill
{\LARGE Huh???? I only gave 2} 

\vfill
{\Large (``self`` is implicitly passed in...)}
\end{frame} 


%%-------------------------------
\begin{frame}[fragile]{LAB}

\vfill
{\Large Let's say you need to render some html...}

\vfill
{\Large The goal is to build a set of classes that render an html page:
\verb|sample_html.html|
}

\vfill
{\Large We'll start with a single class, then add some sub-classes to specialize the behavior}

\vfill
More details in \verb|week-06/LAB_instuctions.rst(html)|
\end{frame}

%%-------------------------------
\begin{frame}[fragile]{LAB}

\vfill
{\Large Step 1:}

\begin{itemize}
  \item Create an "Element" class for rendering an html element (xml element). 
  \item It should have class attributes for the tag name  and the
  indentation
  \item the constructor signature should look like:
    \verb|Element(content=None)| where content is a string
  \item It should have an "append" method that can add another string to the content
  \item It should have a \verb|render(file_out, ind = "")| method that renders the tag and the strings in the content.

     \verb|file_out| could be any file-like object.

     \verb|ind| is a string with enough spaces to indent properly.
\end{itemize}

\end{frame}

%-------------------------------
\begin{frame}{Lightning Talk}

{\centering

\vfill
{\LARGE Lightning Talk:  }

\vfill
{\Large Maria Petrova}

\vfill
}
\end{frame}

\section{Subclassing/Inheritance}

% ---------------------------------------------
\begin{frame}[fragile]{Inheritance}

In object-oriented programming (OOP), inheritance is a way to reuse code of
existing objects, or to establish a subtype from an existing object.

\vfill
...

\vfill
objects are defined by classes, classes can inherit attributes and behavior
from pre-existing classes called base classes, or super classes.

\vfill
The resulting classes are known as derived classes or subclasses.

\vfill
(\url{http://en.wikipedia.org/wiki/Inheritance_%28object-oriented_programming%29})
\end{frame} 

% ---------------------------------------------
\begin{frame}[fragile]{Subclassing}

A subclass ``inherits'' all the attributes (methods, etc) of the parent class.

\vfill
You can then change (``override'') some or all of the attributes to change the behavior.

\vfill
The simplest subclass in Python:

\begin{verbatim}
class A_Subclass(The_SuperClass):
    pass
\end{verbatim}

\vfill
\verb|A_subclass| now has exactly the same behavior as \verb|The_SuperClass|

\end{frame} 

% ---------------------------------------------
\begin{frame}[fragile]{Overriding attributes}

{\Large Overriding is as simple as creating a new attribute with the same name:}

\vfill
\begin{verbatim}
class Circle(object):
    color = "red"
...
class NewCircle(Circle):
    color = "blue"
>>> nc = NewCircle
>>> print nc.color
blue
\end{verbatim}

\vfill
all the \verb|self| instances will have the new attribute
\end{frame} 

% ---------------------------------------------
\begin{frame}[fragile]{Overriding methods}

{\Large Same thing, but with methods}

\vfill
\begin{verbatim}
class Circle(object):
...
    def grow(self, factor=2):
        """grows the circle's diameter by factor"""
        self.diameter = self.diameter * factor
...
class NewCircle(Circle):
...
    def grow(self, factor=2):
        """grows the area by factor..."""
        self.diameter = self.diameter * math.sqrt(2)
\end{verbatim}
all the instances will have the new method
\end{frame} 

\begin{frame}

{\Large
``Here's a program design suggestion: whenever you override a method, the
interface of the new method should be the same as the old.  It should take
the same parameters, return the same type, and obey the same preconditions
and postconditions.  If you obey this rule, you will find that any function
designed to work with an instance of a superclass, like a Deck, will also work
with instances of subclasses like a Hand or PokerHand.  If you violate this
rule, your code will collapse like (sorry) a house of cards.''
}
\vfill
\hfill ThinkPython 18.10
\end{frame}

%%-------------------------------
\begin{frame}[fragile]{LAB}

\vfill
{\Large Step 2:}

\begin{itemize}
  \item  Create a couple subclasses of \verb|Element|, for a \verb|<body>| tag
         and \verb|<p>| tag. Simply override the \verb|tag| class attribute.
  \item Extend the \verb|Element.render()| method so that it can render other
        elements inside the tag in addition to strings. Simple recursion should
        do it. i.e. it can call the \verb|render()| method of the elements it
        contains.
  \item Deal with the content items that could be either simple strings or
        \verb|Element|s with \verb|render| methods...there are a few ways to handle that...
\end{itemize}

\end{frame}


%%-------------------------------
\begin{frame}[fragile]{LAB}

\vfill
{\Large Step 3:}

\begin{itemize}
  \item Create a \verb|<head>| element -- simple subclass.
  \item Create a \verb|OneLineTag| subclass of Element:
        It should override the render method, to render everything on one line --
        for the simple tags, like:
    
        \verb|<title> PythonClass - Class 6 example </title>|
  \item Create a Title subclass of \verb|OneLineTag| class for the title.
  
  \item You should now be able to render an html doc with a head element, with
       a \verb|title| element in that, and a body element with some \verb|<P>|
       elements and some text.
\end{itemize}

\end{frame}

%-------------------------------
\begin{frame}[fragile]{Homework}

{\LARGE Catch Up!}

\vfill
{\Large Read up on OO if you haven't already}

\vfill
{\Large Finish today's Lab}

\vfill
{\Large Finish other Homework / Labs you may not have gotten to.}

\vfill
{\Large Come up with a project proposal}

\end{frame}


\end{document}

 
