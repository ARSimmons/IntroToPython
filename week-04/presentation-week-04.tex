\documentclass{beamer}
%\usepackage[latin1]{inputenc}
\usetheme{Warsaw}
\title[Intro to Python: Week 4]{Introduction  to Python\\ Dictionaries, Sets, Exceptions\\ Files and Text Processing}
\author{Christopher Barker}
\institute{UW Continuing Education}
\date{October 22, 2013}

\usepackage{listings}
\usepackage{hyperref}

\begin{document}

% ---------------------------------------------
\begin{frame}
  \titlepage
\end{frame}

% ---------------------------------------------
\begin{frame}
\frametitle{Table of Contents}
%\tableofcontents[currentsection]
  \tableofcontents
\end{frame}


\section{Review/Questions}

% ---------------------------------------------
\begin{frame}{Review of Previous Class}

\begin{itemize}
  \item Sequences
  \item Lists
  \item Tuples
\end{itemize}

\vfill
{\Large Any questions?}

\end{frame}


% ---------------------------------------------
\begin{frame}{Lightning Talks}

\vfill
{\LARGE Lightning talks today:}

\vfill
{\Large ( Jo-Anne Antoun )}

\vfill
{\Large
 Sako Eaton

\vfill
Brandon Ivers

\vfill
Gary Pei

\vfill
Nathan Savage

}
\vfill

\end{frame}


% --------------------------------------------
\begin{frame}[fragile]{Notes on Workflow}

  \vfill
  {\Large For more than a few lines:}

  \vfill
  {\large Write your code in a module}

  \vfill
  {\large Have a way to re-run quickly}
  \begin{itemize}
    \item Plain command line: \verb|$ python my_script.py|
    \item iPython: \verb|run my_script.py|
    \item The ``run'' button / keystroke in your IDE.
  \end{itemize}

  \vfill

\end{frame}


% ---------------------------------------------
\begin{frame}{Finish Last Class...}

  \vfill
  {\Large More on Looping}

  \vfill
  {\Large Strings!}

  \vfill

\end{frame}

% ##################################
\section{Dictionaries and Sets}

% ---------------------------------------------
\begin{frame}[fragile]{Dictionary}

{\Large Python calls it a \verb|dict| }

\vfill
{\Large Other languages call it:}
\begin{itemize}
  \item dictionary
  \item associative array
  \item map
  \item hash table
  \item hash
  \item key-value pair
\end{itemize}

\vfill

\end{frame} 

% ---------------------------------------------
\begin{frame}[fragile]{Dictionary Constructors}

\begin{verbatim}
>>> {'key1': 3, 'key2': 5}
{'key1': 3, 'key2': 5}

>>> dict([('key1', 3),('key2', 5)])
{'key1': 3, 'key2': 5}

>>> dict(key1=3, key2= 5)
{'key1': 3, 'key2': 5}

>>> d = {}
>>> d['key1'] = 3
>>> d['key2'] = 5
>>> d
{'key1': 3, 'key2': 5}
\end{verbatim}
% {\Large Which to use depends on the shape of your data}

\end{frame} 

% ---------------------------------------------
\begin{frame}[fragile]{Dictionary Indexing}

\begin{verbatim}
>>> d = {'name': 'Brian', 'score': 42}
>>> d['score']
42
>>> d = {1: 'one', 0: 'zero'}
>>> d[0]
'zero'
>>> d['non-existing key']
Traceback (most recent call last):
  File "<stdin>", line 1, in <module>
KeyError: 'non-existing key'
\end{verbatim}

\end{frame} 

% ---------------------------------------------
\begin{frame}[fragile]{Dictionary Indexing}

{\Large Keys can be any immutable:}
\begin{itemize}
  \item numbers
  \item string
  \item tuples
\end{itemize}

\begin{verbatim}
In [325]: d[3] = 'string'
In [326]: d[3.14] = 'pi'
In [327]: d['pi'] = 3.14
In [328]: d[ (1,2,3) ] = 'a tuple key'
In [329]: d[ [1,2,3] ] = 'a list key'
   TypeError: unhashable type: 'list'
\end{verbatim}

\vfill
Actually -- any "hashable" type.
\end{frame} 

% ---------------------------------------------
\begin{frame}[fragile]{Dictionary Indexing}

\vfill
{\Large hash functions convert arbitrarily large data to a small proxy (usually int)

\vfill
always return the same proxy for the same input

\vfill
MD5, SHA, etc
\vfill
}
\end{frame} 

% ---------------------------------------------
\begin{frame}[fragile]{Dictionary Indexing}

\vfill
{\Large
Dictionaries hash the key to an integer proxy and use it to find the key and value
}
\vfill
{\Large
Key lookup is efficient because the hash function leads directly to a bucket with a very few keys (often just one)
}
\vfill
\end{frame} 

% ---------------------------------------------
\begin{frame}[fragile]{Dictionary Indexing}

\vfill
{\Large
What would happen if the proxy changed after storing a key?
}
\vfill
{\Large
Hashability requires immutability}
\vfill
\end{frame} 

% ---------------------------------------------
\begin{frame}[fragile]{Dictionary Indexing}

\vfill
{\Large

Key lookup is very efficient

\vfill
Same average time regardless of size
}

\vfill
also... Python name look-ups are implemented with dict:

 --- its highly optimized
\end{frame} 


% ---------------------------------------------
\begin{frame}[fragile]{Dictionary Indexing}

\vfill
{\Large
{\center 

key to value

lookup is one way

}}
\vfill
{\Large
{\center 

value to key

requires visiting the whole dict

}}

\vfill
{\Large
if you need to check dict values often, create another dict or set (up to you to keep them in sync)

}
\vfill
\end{frame} 

% ---------------------------------------------
\begin{frame}[fragile]{Dictionary Ordering (not)}

\vfill
{\Large
dictionaries have no defined order
}
\vfill
\begin{verbatim}
In [352]: d = {'one':1, 'two':2, 'three':3}

In [353]: d
Out[353]: {'one': 1, 'three': 3, 'two': 2}

In [354]: d.keys()
Out[354]: ['three', 'two', 'one']
\end{verbatim}
\vfill
\end{frame} 

%-------------------------------
\begin{frame}[fragile]{Dictionary Iterating}

{\Large \verb|for| iterates the keys}
\vfill
\begin{verbatim}
>>> d = {'name': 'Brian', 'score': 42}
>>> for x in d:
...   print x
...
score name
\end{verbatim}
\vfill
{note the different order...}
\end{frame}

%-------------------------------
\begin{frame}[fragile]{dict keys and values}

\vfill
\begin{verbatim}
>>> d.keys()
['score', 'name']

>>> d.values()
[42, 'Brian']

>>> d.items()
[('score', 42), ('name', 'Brian')]
\end{verbatim}
\vfill
\end{frame}

%-------------------------------
\begin{frame}[fragile]{dict keys and values}

{\Large iterating on everything}
\vfill
\begin{verbatim}
>>> d = {'name': 'Brian', 'score': 42}
>>> for k, v in d.items():
...   print "%s: %s" % (k, v)
...
score: 42
name: Brian
\end{verbatim}
\vfill
\end{frame}

% ---------------------------------------------
\begin{frame}[fragile]{Dictionary Performance }

\begin{itemize}
  \item indexing is fast and constant time: O(1)
  \item x in s cpnstant time: O(1)
  \item visiting all is proportional to n: O(n)
  \item inserting is constant time: O(1)
  \item deleting is constant time: O(1)
\end{itemize}

\vfill
\url{ http://wiki.python.org/moin/TimeComplexity}

\end{frame} 



% ---------------------------------------------
\begin{frame}[fragile]{ Sets }

\vfill
{\Large \verb|set| is an unordered collection of distinct values}

\vfill
{\Large Essentially a dict with only keys}

\vfill

\end{frame} 

%-------------------------------
\begin{frame}[fragile]{Set Constructors}

\vfill
\begin{verbatim}
>>> set()
set([])
>>> set([1, 2, 3])
set([1, 2, 3])
# as of 2.7
>>> {1, 2, 3}
set([1, 2, 3])
>>> s = set()
>>> s.update([1, 2, 3])
>>> s
set([1, 2, 3])
\end{verbatim}
\vfill

\end{frame}

% ---------------------------------------------
\begin{frame}[fragile]{ Set Properties}

\vfill
{\Large \verb|Set| members must be hashable}

\vfill
{\Large Like dictionary keys -- and for same reason (efficient lookup)}

\vfill
{\Large No indexing (unordered) }

\vfill
\begin{verbatim}
>>> s[1]
Traceback (most recent call last):
  File "<stdin>", line 1, in <module>
TypeError: 'set' object does not support indexing
\end{verbatim}

\vfill
\end{frame} 

% ---------------------------------------------
\begin{frame}[fragile]{ Set Methods}

\begin{verbatim}
>> s = set([1])
>>> s.pop() # an arbitrary member
1
>>> s.pop()
Traceback (most recent call last):
  File "<stdin>", line 1, in <module>
KeyError: 'pop from an empty set'

>>> s = set([1, 2, 3])
>>> s.remove(2)
>>> s.remove(2)
Traceback (most recent call last):
  File "<stdin>", line 1, in <module>
KeyError: 2
\end{verbatim}

\vfill
\end{frame} 

% ---------------------------------------------
\begin{frame}[fragile]{ Set Methods}

\begin{verbatim}
s.isdisjoint(other)

s.issubset(other)

s.union(other, ...)

s.intersection(other, ...)

s.difference(other, ...)

s.symmetric_difference( other, ...)
\end{verbatim}

\vfill
\end{frame} 

% ---------------------------------------------
\begin{frame}[fragile]{ Frozen Set}

\vfill
{\Large Also \verb|frozenset|}

\vfill
{\Large immutable -- for use as a key in a dict\\
(or another set...)}

\vfill
\begin{verbatim}
>>> fs = frozenset((3,8,5))
>>> fs.add(9)
Traceback (most recent call last):
  File "<stdin>", line 1, in <module>
AttributeError: 'frozenset' object has no attribute 'add'
\end{verbatim}

\vfill
\end{frame} 

%-------------------------------
\begin{frame}[fragile]{LAB}

\vfill
{\Large Dictionary LAB:}

\vfill
{\large \verb|code/dict_lab.html (rst) |}

\vfill
\end{frame}


%-------------------------------
\begin{frame}{Lightning Talks}

{\LARGE Lightning Talks:}

\vfill
{\Large
 Sako Eaton

\vfill
Brandon Ivers
}
\vfill

\end{frame}

%%%%%%%%%%%%%%%%%%%%%%%%%
\section{Exceptions}

%-----------------------------------
\begin{frame}[fragile]{Exceptions}

{\Large Another Branching structure:}
\vfill
\begin{verbatim}
try:
    do_something()
    f = open('missing.txt')
    process(f)   # never called if file missing
except IOError:
    print "couldn't open missing.txt"
\end{verbatim}
\vfill
\end{frame}

\begin{frame}[fragile]{Exceptions}

{\Large Never Do this:}
\vfill
\begin{verbatim}
try:
    do_something()
    f = open('missing.txt')
    process(f)   # never called if file missing
except:
    print "couldn't open missing.txt"
\end{verbatim}
\vfill
\end{frame}

\begin{frame}[fragile]{Exceptions}

{\Large Use Exceptions, rather than your own tests

 \hspace{0.1in} -- Don't do this:}

\vfill
\begin{verbatim}
do_something()
if os.path.exists('missing.txt'):
    f = open('missing.txt')
    process(f)   # never called if file missing
\end{verbatim}
\vfill
it will almost always work -- but the almost will drive you crazy
\end{frame}


\begin{frame}[fragile]{Exceptions}

{\centering

{\Large "easier to ask forgiveness than permission"
\vfill
\hfill -- Grace Hopper
}
}

\vfill
\url{http://www.youtube.com/watch?v=AZDWveIdqjY}

(Pycon talk by Alex Martelli)
\end{frame}


\begin{frame}[fragile]{Exceptions}

\vfill
{\Large 
For simple scripts, let exceptions happen\\
\vfill

Only handle the exception if the code can and will do something about it
}
\vfill
(much better debugging info when an error does occur)
\end{frame}


\begin{frame}[fragile]{Exceptions -- finally }

\vfill
\begin{verbatim}
try:
    do_something()
    f = open('missing.txt')
    process(f)   # never called if file missing
except IOError:
    print "couldn't open missing.txt"
finally:
    do_some_clean-up
\end{verbatim}
\vfill
{\Large the \verb|finally:| clause will always run}
\end{frame}

\begin{frame}[fragile]{Exceptions -- else }

\vfill
\begin{verbatim}
try:
    do_something()
    f = open('missing.txt')
except IOError:
    print "couldn't open missing.txt"
else:
    process(f) # only called if there was no exception
\end{verbatim}
\vfill
{\Large Advantage:

you know where the Exception came from}
\end{frame}

%--------------------------------------------
\begin{frame}[fragile]{Exceptions -- using them }

\vfill
\begin{verbatim}
try:
    do_something()
    f = open('missing.txt')
except IOError as the_error:
    print the_error
    the_error.extra_info = "some more information"
    raise
\end{verbatim}

{\Large Particularly useful if you catch more than one exception:}

\begin{verbatim}
except (IOError, BufferError, OSError) as the_error:
    do_something_with (the_error)
\end{verbatim}

\end{frame}


\begin{frame}[fragile]{Raising Exceptions }

\begin{verbatim}
def divide(a,b):
    if b == 0:
        raise ZeroDivisionError("b can not be zero")
    else:
        return a / b
\end{verbatim}
\vfill
{\Large when you call it: }
\vfill
\begin{verbatim}
In [515]: divide (12,0)

ZeroDivisionError: b can not be zero
\end{verbatim}

\end{frame}



\begin{frame}[fragile]{Built in Exceptions}

{\Large You can create your own custom exceptions}

{\Large But...}

\begin{verbatim}
exp = \
 [name for name in dir(__builtin__) if "Error" in name]

len(exp)
32
\end{verbatim}

{\Large For the most part, you can/should use a built in one}

\end{frame}


%-------------------------------
\begin{frame}[fragile]{LAB}

{\Large Exceptions Lab: Improving \verb|raw_input|:}

{\large
\vfill
The \verb|raw_input()| function can generate two exceptions:
\verb|EOFError| or \verb|KeyboardInterrupt| on end-of-file
(EOF) or canceled input.

\vfill
Create a wrapper function, perhaps \verb|safe_input()| that returns
\verb|None| rather rather than raising these exceptions, when
the user enters \verb|^C| for Keyboard Interrupt, or \verb|^D|
(\verb|^Z| on Windows) for End Of File.
}

\vfill
\end{frame}

%-------------------------------
\begin{frame}{Lightning Talks}

{\LARGE Lightning Talks:}

{\Large
\vfill
Gary Pei

\vfill
Nathan Savage
}
\vfill

\end{frame}

\section{File Reading and Writing}

%-------------------------------
\begin{frame}[fragile]{Files}

{\Large Text Files}

\begin{verbatim}
f = open('secrets.txt')
secret_data = f.read()
f.close()
\end{verbatim}

{\Large \verb|secret_data| is a string}

\vfill
(can also use \verb|file()| -- \verb|open()| is preferred)
\end{frame}

%-------------------------------
\begin{frame}[fragile]{Files}

{\Large Binary Files}

\begin{verbatim}
f = open('secrets.txt', 'rb')
secret_data = f.read()
f.close()
\end{verbatim}

{\Large \verb|secret_data| is still a string \\[.1in]
(with arbitrary bytes in it)}
\vfill
(See the \verb|struct| module to unpack binary data )
\end{frame}

%-------------------------------
\begin{frame}[fragile]{Files}

{\Large File Opening Modes}
\vfill
\begin{verbatim}
f = open('secrets.txt', [mode])

'r', 'w', 'a'
'rb', 'wb', 'ab'
r+, w+, a+
r+b, w+b, a+b
U
U+
\end{verbatim}
\vfill
{\Large Gotcha -- w mode always clears the file}
\end{frame}

%-------------------------------
\begin{frame}[fragile]{Text File Notes}

{\Large Text is default}
\begin{itemize}
  \item Newlines are translated: \verb|\r\n -> \n|
  \item   -- reading and writing!
  \item Use *nux-style in your code: \verb|\n|
  \item Open text files with \verb|'U'| "Universal" flag
\end{itemize}

\vfill
{\Large Gotcha:}
\begin{itemize}
  \item  no difference between text and binary on *nix\\
  \begin{itemize}
    \item breaks on Windows
  \end{itemize}
\end{itemize}

\end{frame}

%-------------------------------
\begin{frame}[fragile]{File Reading}

{\Large Reading Part of a file}

\begin{verbatim}
header_size = 4096

f = open('secrets.txt')
secret_data = f.read(header_size)
f.close()
\end{verbatim}

\end{frame}

%-------------------------------
\begin{frame}[fragile]{File Reading}

{\Large Common Idioms}

\begin{verbatim}
for line in open('secrets.txt'):
    print line
\end{verbatim}

\begin{verbatim}
f = open('secrets.txt')
while True:
    line = f.readline()
    if not line: 
        break
    do_something_with_line()
\end{verbatim}

\end{frame}

%-------------------------------
\begin{frame}[fragile]{File Writing}

\begin{verbatim}

outfile = open('output.txt', 'w')

for i in range(10):
    outfile.write("this is line: %i\n"%i)

\end{verbatim}

\end{frame}

%-------------------------------
\begin{frame}[fragile]{File Methods}

{\Large Commonly Used Methods}
\begin{verbatim}

f.read() f.readline()  f.readlines() 

f.write(str) f.writelines(seq)
 
f.seek(offset)   f.tell()

f.flush()           
           
f.close() 
\end{verbatim}

\end{frame}

%-------------------------------
\begin{frame}[fragile]{File Like Objects}

{\Large File-like objects }
\vfill
{\large Many classes implement the file interface:}
\vfill
\begin{itemize}
  \item loggers
  \item \verb|sys.stdout|
  \item \verb|urllib.open()|
  \item pipes, subprocesses
  \item StringIO
\end{itemize}

\url{http://docs.python.org/library/stdtypes.html#bltin-­‐file-­‐objects}
\end{frame}

%-------------------------------
\begin{frame}[fragile]{StringIO}

{\Large StringIO }
\vfill
\begin{verbatim}
In [417]: import StringIO
In [420]: f = StringIO.StringIO()

In [421]: f.write("somestuff")

In [422]: f.seek(0)

In [423]: f.read()
Out[423]: 'somestuff'
\end{verbatim}

{\Large handy for testing}
\end{frame}




%-------------------------------
\begin{frame}[fragile]{LAB}

\vfill
{\large Some lab excercises}
\vfill

\end{frame}


\section{Paths and Directories}

% ----------------------------------
\begin{frame}[fragile]{Paths}

{\Large Relative paths:}

\begin{verbatim}
secret.txt
./secret.txt
\end{verbatim}

{\Large Absolute paths:}
\begin{verbatim}
/home/chris/secret.txt
\end{verbatim}

{\Large Either work with \verb|open()|, etc.}

\vfill
(working directory only makes sense with command-line programs...)
\end{frame}

% ----------------------------------
\begin{frame}[fragile]{os.path}

\begin{verbatim}
os.getcwd() -- os.getcwdu()
chdir(path)

os.path.abspath()
os.path.relpath()
\end{verbatim}

\end{frame}

% ----------------------------------
\begin{frame}[fragile]{os.path}

\vfill
\begin{verbatim}
os.path.split()
os.path.splitext()
os.path.basename()
os.path.dirname()
os.path.join()
\end{verbatim}

\vfill
(all platform independent)

\end{frame}


% ----------------------------------
\begin{frame}[fragile]{directories}

\vfill
\begin{verbatim}
os.listdir()
os.mkdir()

os.walk()

\end{verbatim}

\vfill
(higher level stuff in \verb|shutil| module)

\end{frame}

%-------------------------------
\begin{frame}[fragile]{LAB}

\begin{itemize}
  \item write a program which prints the full path to all files
    in the current directory, one per line
  \item write a program which copies a file from a source, to a
        destination (without using shutil, or the OS copy command)
  \item write a program that extracts all the programming languages that the students in this class used before (\verb|code\students_languages.txt|) 
  \item update mail-merge from the earlier lab to write output
         to individual files on disk
\end{itemize}

\end{frame}


%-------------------------------
\begin{frame}[fragile]{Homework}

{\large Recommended Reading}
\begin{itemize}
  \item Dive Into Python: Chapt. 13,14
  \item Unicode: \url{http://www.joelonsoftware.com/articles/Unicode.html}
\end{itemize}

\vfill
{\large Do the Labs you didn't finish in class}

\vfill
\begin{itemize}
  \item Coding Kata 14 - Dave Thomas \\
    \url{http://codekata.pragprog.com/2007/01/ kata_fourteen_t.html}

  \item Use The Adventures of Sherlock Holmes as input:\\
        \verb|code/sherlock.txt| (ascii)

  \item  This is intentionally open-ended and underspecified. There are many interesting decisions to make.

  \item Experiment with different lengths for the lookup key. (3 words, 4 words, 3 letters, etc)
\end{itemize}

\end{frame}


\end{document}

 
