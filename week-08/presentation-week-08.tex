\documentclass{beamer}
%\usepackage[latin1]{inputenc}
\usetheme{Warsaw}
\title[Intro to Python: Week 8]{Introduction  to Python\\ More OO}
\author{Christopher Barker}
\institute{UW Continuing Education}
\date{November 19, 2013}

\usepackage{listings}
\usepackage{hyperref}

\begin{document}

% ---------------------------------------------
\begin{frame}
  \titlepage
\end{frame}

% ---------------------------------------------
\begin{frame}
\frametitle{Table of Contents}
%\tableofcontents[currentsection]
  \tableofcontents
\end{frame}


\section{Review/Questions}

% ---------------------------------------------
\begin{frame}{Review of Previous Class}

\begin{itemize}
  \item .
  \item .
  \item .
\end{itemize}

\end{frame}


% ---------------------------------------------
\begin{frame}{Lightning Talks}

\vfill
{\LARGE Lightning talks today:}

\vfill
{\Large
 

\vfill

\vfill

\vfill

}
\vfill

\end{frame}


% ---------------------------------------------
\begin{frame}{Homework review}

  \vfill
  {\Large Homework Questions? }

  \vfill
  {\Large My Solution}

  \vfill

\end{frame}

%%%%%%%%%%%%%%%%%%%%%%%%%%%%%%%%%%%%
\section{Special Methods}

\begin{frame}[fragile]{special methods}

{\Large Python's Duck typing:}

\vfill
{\Large Defining special (or magic) methods in your classes is how you make
your class act like standard classes}

\end{frame} 

\begin{frame}[fragile]{special methods}

{\Large We've seen at least one:}

\begin{verbatim}
__init__
\end{verbatim}

\vfill
{\Large it's all in the double underscores...}

\vfill
{\Large Pronounced ``dunder'' (or ``under-under'') }

\vfill
{\Large try: \verb|dir(2)| or \verb|dir(list)|}

\end{frame} 

\begin{frame}[fragile]{special methods}

{\Large Emulating Numeric types}

\begin{verbatim}
object.__add__(self, other)
object.__sub__(self, other)
object.__mul__(self, other)
object.__floordiv__(self, other)
object.__mod__(self, other)
object.__divmod__(self, other)
object.__pow__(self, other[, modulo])
object.__lshift__(self, other)
object.__rshift__(self, other)
object.__and__(self, other)
object.__xor__(self, other)
object.__or__(self, other)¶
\end{verbatim}

\end{frame} 

\begin{frame}[fragile]{special methods}

{\Large Emulating container types:}

\begin{verbatim}
object.__len__(self)
object.__getitem__(self, key)
object.__setitem__(self, key, value)
object.__delitem__(self, key)
object.__iter__(self)
object.__reversed__(self)
object.__contains__(self, item)
object.__getslice__(self, i, j)
object.__setslice__(self, i, j, sequence)
object.__delslice__(self, i, j)
\end{verbatim}

\end{frame} 

\begin{frame}[fragile]{special methods}

{\Large Example -- to define addition:}

\begin{verbatim}
def __add__(self, v):
    """
    redefine + as element-wise vector sum
    """
    assert len(self) == len(v)
    return vector([x1 + x2 for x1, x2 in zip(self, v)])
\end{verbatim}

( from a nice complete example in \verb|code/vector.py| )

\end{frame} 


\begin{frame}[fragile]{special methods}

{\Large You get the idea...}

\vfill
{\Large You only need to define the ones that are going to get used}

\vfill
{\Large But you probably want to define at least these:}

\vfill
\verb|object.__str__|: Called by the str() built-in function and by the print statement to compute the “informal” string representation of an object.

\vfill
\verb|object.__repr__|: Called by the repr() built-in function and by string conversions (reverse quotes) to compute the “official” string representation of an object.

\end{frame} 

\begin{frame}[fragile]{special methods}

\vfill
{\Large When you want your class to act like a "standard" class in some way:}

\vfill
{\Large Look up the magic methods you need and define them}

\vfill
\url{http://docs.python.org/reference/datamodel.html#special-method-names}

\vfill
\url{http://www.rafekettler.com/magicmethods.html}
\end{frame} 

\begin{frame}[fragile]{LAB}

{\Large Write a ``Circle'' class:}


\vfill
{\large A Circle has a radius and can compute its area:}
\begin{verbatim}
In [2]: c = Circle(3)
In [3]: c.radius
Out[3]: 3
In [4]: c.get_area()
Out[4]: 28.274333882308138
In [5]: print c
Circle Object with radius: 3.000000
\end{verbatim}
{\large Write an \verb|__add__| method so you can add two circles }

\vfill
{\large Have \verb|__str__| and \verb|__repr__| methods }

\vfill
{\large Extra credit: also compare them... (\verb|c1 > c2|, etc)}

\vfill
{\large \verb|code/circle.py| and \verb|code/test_circle.py|}
\end{frame}


%-------------------------------
\begin{frame}{LAB}

\vfill
{\large Some lab excercises}
\vfill

\end{frame}


%-------------------------------
\begin{frame}{Lightning Talk}

{\LARGE Lightning Talks:}

\vfill
{\large person 1}

\vfill
{\large person 2}

\end{frame}



%-------------------------------
\begin{frame}[fragile]{Wrap Up}

{\LARGE Thinking OO in Python:}

\vfill
{\large Think about what makes sense for your code:}
\begin{itemize}
  \item {\large Code re-use}
  \item {\large Clean APIs}
  \item {\large ... }
\end{itemize}

\vfill
{\large Don't be a slave to what OO is \emph{supposed} to look like. }

\vfill
{\large Let OO work for you, not \emph{create} work for you}

\end{frame}


%-------------------------------
\begin{frame}[fragile]{Wrap Up}

{\Large OO in Python:}

\vfill
{\Large The Art of Subclassing}: Raymond Hettinger

\vfill
{\small \url{http://pyvideo.org/video/879/the-art-of-subclassing}}

\vfill
''classes are for code re-use -- not creating taxonomies''

\vfill
{\Large Stop Writing Classes}: Jack Diederich

\vfill
{\small \url{http://pyvideo.org/video/880/stop-writing-classes}}

\vfill
``If your class has only two methods -- and one of them is \verb|__init__|
-- you don't need a class ''
\end{frame}




%-------------------------------
\begin{frame}[fragile]{Homework}

{\Large Finish the labs.}

{\Large You should have a good start on your project by the end of this week}

\end{frame}
 




%-------------------------------
\begin{frame}[fragile]{Homework}

Recommended Reading:
\begin{itemize}
  \item some stuff
\end{itemize}

Do:
\begin{itemize}
    \item Some things    
\end{itemize}

\end{frame}


\end{document}

 
